% ----------------------------------------------------------
% Introdução 
% Capítulo sem numeração, mas presente no Sumário
% ----------------------------------------------------------

\chapter[Introdução]{Introdução}
\addcontentsline{toc}{chapter}{Introdução}

Segundo TUDE, o enlace de rádio pode ser definido como : "
Uma aplicação da transmissão de informação por meio de ondas eletromagnéticas, se caracterizando como uma das aplicação que faz parte das Segundo Tude , “Um enlace rádio digital ponto a ponto é utilizado para o transporte de informação entre dois pontos fixos,tendo o espaço livre como meio de transmissão (wireless)”.[1]


\section{Proposta}\label{sec:motivacao}
Desenvolver aplicação no software Elipse E3, que se comunica com a placa microcontrolada Arduino, utilizando os protocolos de comunicação MODBUS RTU e MODBUS TCP.

\subsection{Requisitos do Projeto}
\begin{itemize}
\item Apresentar uma interface gráfica no Sotfware Elipse E3 para comunicação com o usuário e display de informações
\item Receber um valor de um número decimal, converter para binário e mostrar no sistema de LEDS e na interface;
\item Apresentar status de botão conectado ao Arduino;
\item Fazer leitura do valor de um potênciometro, adicionar um \textit{offset} fornecido pelo usuário e apresentar na interface;
\end{itemize}
\section{Fundamentação teórica}
Para realização do trabalho foram necessários os seguintes conhecimentos relativos à redes e comunicações.
\subsection{Redes Industriais}]
migue aplicações
\subsection{Comunicação Serial}
Comunicação serial é a nomenclatura atribuída ao processo da troca de dados de forma sequencial por meio de um canal ou barramento específico para comunicação . Tais dados são bytes de informação, transmitidos bit a bit, por meio de uma porta serial.	
Os protocolos de comunicação serial representam o modo que a mensagem é transmitida, definindo a forma como os bytes serão ordenados de modo a garantir que a mensagem seja transmitida.
O padrão de comunicação diz respeito à estrutura física da comunicação, fazendo referência aos padrões elétricos envolvidos e quantidade de vias utilizadas.
\subsection{Protocolo MODBUS}
\subsubsection{RTU}
migue aplicações
\subsubsection{TCP}
